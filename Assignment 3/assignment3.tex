%%%%%%%%%%%%%%%%%%%%%%%%%%%%%%%%%%%%%%%%%
%
% CMPT 435
% Lab Zero
%
%%%%%%%%%%%%%%%%%%%%%%%%%%%%%%%%%%%%%%%%%

%%%%%%%%%%%%%%%%%%%%%%%%%%%%%%%%%%%%%%%%%
% Short Sectioned Assignment
% LaTeX Template
% Version 1.0 (5/5/12)
%
% This template has been downloaded from: http://www.LaTeXTemplates.com
% Original author: % Frits Wenneker (http://www.howtotex.com)
% License: CC BY-NC-SA 3.0 (http://creativecommons.org/licenses/by-nc-sa/3.0/)
% Modified by Alan G. Labouseur  - alan@labouseur.com, and Patrick Tyler - Patrick.Tyler1@marist.edu
%
%%%%%%%%%%%%%%%%%%%%%%%%%%%%%%%%%%%%%%%%%

%----------------------------------------------------------------------------------------
%	PACKAGES AND OTHER DOCUMENT CONFIGURATIONS
%----------------------------------------------------------------------------------------

\documentclass[letterpaper, 10pt]{article} 

\usepackage[english]{babel} % English language/hyphenation
\usepackage{graphicx}
\usepackage[lined,linesnumbered,commentsnumbered]{algorithm2e}
\usepackage{listings}
\usepackage{fancyhdr} % Custom headers and footers
\pagestyle{fancyplain} % Makes all pages in the document conform to the custom headers and footers
\usepackage{lastpage}
\usepackage{xcolor}
\usepackage{url}
\usepackage{titlesec}

% Stolen from https://www.overleaf.com/learn/latex/Code_listing 
\definecolor{codegreen}{rgb}{0,0.6,0}
\definecolor{codegray}{rgb}{0.5,0.5,0.5}
\definecolor{codepurple}{rgb}{0.58,0,0.82}
\definecolor{backcolour}{rgb}{0.95,0.95,0.92}

\lstdefinestyle{mystyle}{
    backgroundcolor=\color{backcolour},   
    commentstyle=\color{codegreen},
    keywordstyle=\color{magenta},
    numberstyle=\tiny\color{codegray},
    stringstyle=\color{codepurple},
    basicstyle=\ttfamily\footnotesize,
    breakatwhitespace=false,         
    breaklines=true,                 
    captionpos=b,                    
    keepspaces=true,                 
    numbers=left,                    
    numbersep=5pt,                  
    showspaces=false,                
    showstringspaces=false,
    showtabs=false,                  
    tabsize=2
}
\lstset{style=mystyle, language=c++}


\fancyhead{} % No page header - if you want one, create it in the same way as the footers below
\fancyfoot[L]{} % Empty left footer
\fancyfoot[C]{page \thepage\ of \pageref{LastPage}} % Page numbering for center footer
\fancyfoot[R]{}

\renewcommand{\headrulewidth}{0pt} % Remove header underlines
\renewcommand{\footrulewidth}{0pt} % Remove footer underlines
\setlength{\headheight}{13.6pt} % Customize the height of the header

%----------------------------------------------------------------------------------------
%	TITLE SECTION
%----------------------------------------------------------------------------------------

\newcommand{\horrule}[1]{\rule{\linewidth}{#1}} % Create horizontal rule command with 1 argument of height

\title{	
   \normalfont \normalsize 
   \textsc{CMPT 435 - Fall 2023 - Dr. Labouseur} \\[10pt] % Header stuff.
   \horrule{0.5pt} \\[0.25cm] 	% Top horizontal rule
   \huge Assignment Three -- \LaTeX ~Graphs\\     	    % Assignment title
   \horrule{0.5pt} \\[0.25cm] 	% Bottom horizontal rule
}

\author{Patrick Tyler \\ \normalsize Patrick.Tyler1@marist.edu}

\date{\normalsize\today} 	% Today's date.

\begin{document}
\maketitle % Print the title
\section{Graph Overview}
Graphs are an important structure representing connected items. They are made of vertices and connected through
edges. Graphs can be implemented in various ways such as an adjacency list, a matrix, or linked objects.
Each has space or time advantages depending on the operation. Linked objects are usually favored for
graph traversals because they offer an ergonomic way to traverse nodes. Adjacency lists 
store a graph of a collection of vertices that each have a collection neighbors (the vertices that this vertex can reach).
Matrices store a graph
as a vertex count by vertex count 2D array of marking which, in an unweighted graph, are usually just store
booleans which are ticked on if an connection from the outer vertex exists to the inner vertex. Linked objects
may store information about themselves such as id and a collection of their neighbors; they 
are usually exposed by just a signal origin pointer which is link to the rest of its connected component; if
there are disconnected components (vertices that are not reachable starting at the origin), they may be
reached by some other way and the graph may implement some other mechanism to reach them.
\subsection{Inserting to a Graph}
Adding an item, often called vertex, to graph can be implemented in constant time\(^1\).
Depending on the use case of the graph it can be useful to sore additional information about the vertices
such as mappings to id's to vertex or index and a collection of the vertices themselves.
In the below code example a vertex is inserted into an existing graph to form semantic adjacency list, matrix,
and linked object structures. There could be some optimizations if the length of the graph is known ahead of time.\\
\(^1\) The code example is O(n) because each matrix item existing in the graph need be resized.
Also, vector push back operations may be O(n) if the internal capacity of the array is exceeded and
therefore the items need to be copied to another bigger array. However, most of the time, especially 
with the amount of vertices in the graphs1.txt test file, push backs occur in constant time because
only the current size counter need be incremented.
% addVertex code listing
\lstinputlisting[linerange={367-386}, firstnumber=367]{main.cpp}
Adding the connections, often called edges, to a graph can also be implemented in constant time.
The indices or linked objects of the two vertices can be provided directly or obtained in constant time with
mappings. Then edges can be added through pushing to a collection in the case of adjacency lists and linked objects
or indexed and marked in the case of the matrix.
% addEdge code listing
\lstinputlisting[linerange={389-411}, firstnumber=389]{main.cpp}
\vspace{.25cm}
\hrule
\vspace{.25cm}
\noindent
Girl: \textit{I do not where this relationship is going. It feels undirected.}\\
Guy: \textit{Great! I knew we had a mutual connection.}\\
Girl: \textit{What? That's not what I meant at all.}\\
Guy: \textit{You see I thought you meant that because undirected gra-}\\
Girl: \textit{Nevermind, it's just you. I'm leaving.}\\
\hrule
\vspace{1cm}
\section{Traversing graphs}
As mentioned before, linked objects are the preferred structure for graph traversals in most cases. This may
be problematic with when graphs has disconnected components, but that too can ultimately be solved. The order
in which the vertices within a graph are processed are dictated by the search method. The time complexity
of traversal should be \(O(|v| + |e|)\) or the sum of the vertices and edge cardinality. This is because each
vertex is set from unseen to seen and each edge is considered once (and only traversed when the vertex is unseen).
In the case of disconnected components, there may be additional complexity in the implementation; however, the
time complexity order remains the same because at worst there will only need to be \(|v|\) more is seen checks with
the same amount of edge checks.
\subsection{Depth First Search}
Depth first search processes the vertices by going deep before branching out.
Implementations of this traversal generally use a stack which obtains the order by pushing
each vertex and its neighbors onto the stack marking and processing each and getting the next
vertex to process by popping it out of the stack until all vertices have been processed.
% dfs listing
\subsection{Breadth First Search}
Breadth first search (BSF) processes the vertices by branching out before going deep.
Implementations of this traversal generally use a queue which obtains the order by enqueuing
each vertex and its neighbors onto the queue marking and processing each and getting the next
vertex to process by dequeuing it out of the queue until all vertices have been processed.
% breadth listing
\vspace{.25cm}
\hrule
\vspace{.25cm}
\noindent
\textit{Joke redacted.}\\
\hrule
\vspace{1cm}

\section{Binary Search Tree}
Binary search trees (BST) are a special type of graph where the vertices or sometimes just called nodes
are arranged such that the left and right subtrees all follow a practical ordering. Usually, this ordering
dictates that all elements in the left subtree of a node are strictly less than and all elements in the right
subtree of a node are greater than or equal to.
\subsection{Path}
The "path" of a node is relative to the root. It represents the outcomes of the comparisons of the node values
down the tree. For instance, a node whose path from the root is L R R would be less than the root node and greater
than or less than the next two nodes down the tree. Below is insertion and search code which displays the path of 
the node. Finding a node by its value / its correct position in the tree are very similar tasks. The program
starts at the root and then travels down the tree pick the left or right node to consider next until it finds
the node with the desired value in the case of a search or until it finds the node with a null pointer in the
case of an insert.
Both ideally work in \(O(logn)\) where n is the number of elements in the BST;
however, depending on the insertion order it can at worst be \(O(n)\) when the data structure forms
more of a stick rather than a tree.

% path listing
\lstinputlisting[linerange={419-458}, firstnumber=419]{main.cpp}
\lstinputlisting[linerange={460-480}, firstnumber=460]{main.cpp}
\subsection{In-order Traversal}
In-order traversals are one of three in the common BST family of searches, and they follow the pattern of
left then root then right. In other words, the right subtree of a node is processed before
the root then the root is processed then the right subtree. The time complexity of this operation is 
\(O(n)\) because it only needs to consider each node once. The code below produces the desired order because
the left subtree to called to be processed then once that entire call stack finishes the root is processed (printed out)
then the right subtree.
% in-order listing
\lstinputlisting[linerange={482-487}, firstnumber=482]{main.cpp}
\vspace{.25cm}
\hrule
\vspace{.25cm}
\noindent
Why don't we teach these computer science topics to kids? \textit{It's too graphic.}
\hrule
\vspace{1cm}
\section{Miscellaneous Code Notes}
Display code and file reading code should be commented well and, as always, is available in
main.cpp file for this assignment.

\end{document}
